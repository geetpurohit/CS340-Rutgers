\documentclass{article}

\setlength{\headsep}{0.75 in}
\setlength{\parindent}{0 in}
\setlength{\parskip}{0.1 in}

%=====================================================
% Add PACKAGES Here (You typically would not need to):
%=====================================================

\usepackage[margin=1in]{geometry}
\usepackage{amsmath,amsthm}
\usepackage{fancyhdr}
\usepackage{enumitem}
\usepackage{graphicx}
%=====================================================
% Ignore This Part (But Do NOT Delete It:)
%=====================================================

\theoremstyle{definition}
\newtheorem{problem}{Problem}
\newtheorem*{fun}{Fun with Algorithms}
\newtheorem*{challenge}{Challenge Yourself}
\def\fline{\rule{0.75\linewidth}{0.5pt}}
\newcommand{\finishline}{\vspace{-15pt}\begin{center}\fline\end{center}}
\newtheorem*{solution*}{Solution}
\newenvironment{solution}{\begin{solution*}}{{} \end{solution*}}
\newcommand{\grade}[1]{\hfill{\textbf{($\mathbf{#1}$ points)}}}
\newcommand{\thisdate}{\today}
\newcommand{\thissemester}{\textbf{Rutgers: Spring 2022}}
\newcommand{\thiscourse}{CS 344: Design and Analysis of Computer Algorithms} 
\newcommand{\thishomework}{Number} 
\newcommand{\thisname}{Name} 
\newcommand{\thisextension}{Yes/No} 

\headheight 40pt              
\headsep 20pt
\renewcommand{\headrulewidth}{0pt}
\lhead{\small \textbf{Only for the personal use of students registered in CS 344, Spring 2022 at Rutgers University. Redistribution out of this class is strictly prohibited.}}
\pagestyle{fancy}

\newcommand{\thisheading}{
   \noindent
   \begin{center}
   \framebox{
      \vbox{\vspace{2mm}
    \hbox to 6.28in { \textbf{\thiscourse \hfill \thissemester} }
       \vspace{8mm}
       \hbox to 6.28in { {\Large \hfill  Midterm Exam \#\thishomework \hfill} }
       \vspace{2mm}
         \hbox to 6.28in { { \hfill ~ April 07, 2022 \hfill} }
       \vspace{4mm}
       \hbox to 6.28in { \emph{Name:  \thisname \hfill NetID:  \thisextension}}
      \vspace{2mm}}
      }
   \end{center}
   \bigskip
}

%=====================================================
% Some useful MACROS (you can define your own in the same exact way also)
%=====================================================


\newcommand{\ceil}[1]{{\left\lceil{#1}\right\rceil}}
\newcommand{\floor}[1]{{\left\lfloor{#1}\right\rfloor}}
\newcommand{\prob}[1]{\Pr\paren{#1}}
\newcommand{\expect}[1]{\Exp\bracket{#1}}
\newcommand{\var}[1]{\textnormal{Var}\bracket{#1}}
\newcommand{\set}[1]{\ensuremath{\left\{ #1 \right\}}}
\newcommand{\poly}{\mbox{\rm poly}}

\newcommand{\totalsum}{\textnormal{\texttt{TOTAL-SUM}}}
\newcommand{\maxalg}{\textnormal{\texttt{MAX-ALG}}}
\newcommand{\minalg}{\textnormal{\texttt{MIN-RAND-ALG}}}

%=====================================================
% Fill Out This Part With Your Own Information:
%=====================================================


\renewcommand{\thishomework}{2} %Homework number
\renewcommand{\thisname}{} % Your name
\renewcommand{\thisextension}{} % Pick only one of the two options accordingly

\begin{document}

\thisheading

\vspace{-0.5cm}
\subsection*{Instructions}

\begin{enumerate}
	\item Do not forget to write your name and NetID above, and to sign Rutgers honor pledge below. 
	\item The exam contains $4$ problems worth $100$ points in total \emph{plus} one {extra} credit problem worth $10$ points. 
	\item This is a take-home exam. You have exactly 24 hours to finish the exam, starting from Thursday, April 07, 9am EST until Friday, April 08, 9am EST.
	\item The exam should be done \textbf{individually} and you are not allowed to discuss these questions with anyone else. This includes asking any questions or clarifications
	regarding the exam from other students or posting them publicly on Piazza (\textbf{any inquiry should be posted privately on Piazza -- also no hints or verifying a partial solution will be given during the exam}). You may however consult all
	the materials used in this course (video lectures, notes, textbook, etc.) while writing your solution, but \textbf{no other resources are allowed}.

	\item Remember that you can leave a problem (or parts of it) entirely blank and receive $25\%$ of the grade for that problem (or part). However, this should not  discourage you from attempting a problem if you think 
	you know how to approach it as you will receive partial credit more than $25\%$ if you are on the right track. But keep in mind that if you simply do not know the answer, writing a very wrong answer may lead to $0\%$ credit.
	
	The only \textbf{exception} to this rule is the extra credit problem: you do {not get any credit for leaving the extra credit problem blank}, and there is almost no partial credit on that problem.
	
	\item \textbf{You should always prove the correctness of your algorithm and analyze its runtime.} Also, as a general rule, avoid using complicated pseudo-code and instead explain your algorithm in English. 
	\item You may use any algorithm presented in the class or homeworks as a building block for your solutions. 
\end{enumerate}

\finishline

\paragraph{Rutgers honor pledge:} 

\begin{quote}
\emph{On my honor, I have neither received nor given any unauthorized assistance on this
examination.} 
\end{quote}
\hfill{Signature:\underline{~~~~~~~~~~~~~~~~~~~~~~~~~~~~~~~~~~~~~~~~~~~~~}}

\bigskip

\begin{center}
\begin{tabular}{|c|r|c|}
\hline
Problem. \# & Points & Score \\ \hline\hline
$1$ & 25 & ~~~~~~~~~~~\\  \hline
$2$ & 25 & \\ \hline
$3$ & 25 & \\ \hline
$4$ & 25 & \\ \hline
$5$ & +10 & \\ \hline
Total & $100 + 10$ & \\ \hline
\end{tabular}
\end{center}

\newpage

\begin{problem}\label{basics}~
\begin{enumerate}[label=(\alph*)]
	\item Suppose $G=(V,E)$ is any undirected graph and $(S,V-S)$ is a cut with zero cut edges in $G$. Prove that if we pick two arbitrary vertices $u \in S$ and $v \in V - S$, and add a new edge $(u,v)$, in the resulting graph, there is no cycle that contains the edge $(u,v)$. 
	\grade{12.5}
	
	\begin{solution}

	\end{solution}
	
	\newpage
	\item Suppose $G = (V,E)$ is an undirected graph with weight $w_e$ on each edge $e$. Additionally, suppose that $G$ has an edge $f$ with weight \emph{strictly larger} than all other edges in $G$. 
	Prove that if $f$ belongs to a \emph{cycle} in $G$, then \emph{no} minimum spanning tree (MST) of $G$ can contain the edge $f$. 
	 \grade{12.5}
	 
	 \begin{solution}

	\end{solution}
	
\end{enumerate}
\end{problem}

\newpage


\begin{problem}\label{greedy}
	You have a bag of $m$ cookies and a group of $n$ friends. For each friend $1 \leq i \leq n$, you know the ``greed factor'' of your friend as a number $G[i]$:
	this is the minimum number of cookies you should give to this friend to make them stop complaining. Of course, you would like to find a way to distribute your cookies in a way to \emph{minimize} the number of your friends that are still complaining. 
	
	Design an $O(n\log{n})$ time greedy algorithm to find an assignment of the cookies to your friends that minimizes the number of the friends that are still complaining: 
	recall that a friend $i$ stops complaining if we assign them $c_i \geq G[i]$ cookies. \grade{25}
	
	Assume that the greed factors of the friends are all distinct. The output should be a list of $n$ pairs consisting of a friend identified by their unique greed factor, and the number of cookies assigned to this friend. 

\paragraph{Example:} An example with $m=11$ and $n=6$ is as follows. 
\begin{itemize}
	\item \textbf{Input:} The array of greed factors $G=[6,3,5,2,8,7]$.
	\item \textbf{Output:} A list of pairs with first elements chosen from $G$ to identify the friend and the second elements denoting the number of cookies assigned to them: 
	\[
		[(5,5),(3,3),(2,3),(6,0),(8,0),(7,0)].
	\]
	This means that the friends
	with greed factors $5,3,$ and $2$, each received $5,3,$ and $3$ cookies respectively, while the remaining friends received zero cookies. This reduces the number of friends who are still complaining to three. 
\end{itemize}
\end{problem}

\begin{solution}

\end{solution}
	

\newpage


\begin{problem}\label{search}
	You are given a directed graph $G=(V,E)$ with blue and red edges, and two vertices $s$ and $t$. Design and analyze an algorithm that outputs 
	whether there exists a \emph{walk} from $s$ to $t$ in $G$ that contains an \emph{even} number of red edges; the walk can contain an arbitrary number of blue edges.  \grade{25}
	
	A complete solution consists of three part: the algorithm or reduction (\textbf{10 points}), the proof of correctness (\textbf{10 points}), 
	and the runtime analysis (\textbf{5 points}). 
\end{problem}
\begin{solution}

\end{solution}
	
\newpage


\begin{problem}\label{mst}
	A \textbf{feedback edge set} of an undirected connected graph $G=(V,E)$ is a set of edges $F \subseteq E$ such that any cycle in $G$ has at least one edge in $F$. In other words, removing the edges in $F$ from $G$ 
	leaves the graph $G-F$ without any cycle. Describe and analyze an $O(m\log{m})$ time algorithm to compute the minimum-weight feedback edge set of a given graph $G=(V,E)$ with \emph{positive} weight $w_e $ on each edge $e \in E$. 

	\grade{25}
	
	A complete solution consists of three part: the algorithm or reduction (\textbf{10 points}), the proof of correctness (\textbf{10 points}), 
	and the runtime analysis (\textbf{5 points}). 


	\smallskip
	\emph{Hint:} Observe that the graph $G - F$ should be a spanning tree. (You have already seen the rest of the solution in your homework!) 
\end{problem}

\begin{solution}

\end{solution}
	
\newpage


\begin{problem}[\textbf{Extra Credit}]\label{mst}
	Consider the following different (and less efficient) algorithm for computing an MST of a given undirected and connected graph $G=(V,E)$ with edge weight $w_e$ on each $e \in E$: 
	\begin{enumerate}
		\item Sort the edges in decreasing (non-increasing) order of their weights. 
		\item Let $H = G$ be a copy of the graph $G$. 
		\item For $i=1$ to $m$ (in the sorted ordering of edges): 
		\begin{enumerate}
			\item If removing $e_i$ from $H$ does not make $H$ disconnected, remove $e_i$ from $H$. 
		\end{enumerate}
		\item Return $H$ as a minimum spanning tree of $G$. 
	\end{enumerate}
	Prove the correctness of this algorithm, i.e., that it outputs an MST of any given graph $G$ (we ignore the runtime of this algorithm in this problem). \grade{+10}
\end{problem}

\begin{solution}

\end{solution}
	

\end{document}





