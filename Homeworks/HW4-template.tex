\documentclass{article}

\setlength{\headsep}{0.75 in}
\setlength{\parindent}{0 in}
\setlength{\parskip}{0.1 in}

%=====================================================
% Add PACKAGES Here (You typically would not need to):
%=====================================================

\usepackage{xcolor}
\usepackage[margin=1in]{geometry}
\usepackage{amsmath,amsthm}
\usepackage{fancyhdr}
\usepackage{enumitem}
\usepackage{graphicx}
\usepackage{xspace}
\usepackage{subcaption}
%=====================================================
% Ignore This Part (But Do NOT Delete It:)
%=====================================================

\theoremstyle{definition}
\newtheorem{problem}{Problem}
\newtheorem*{fun}{Fun with Algorithms}
\newtheorem*{challenge}{Challenge Yourself}
\def\fline{\rule{0.75\linewidth}{0.5pt}}
\newcommand{\finishline}{\begin{center}\fline\end{center}}
\newtheorem*{solution*}{Solution}
\newenvironment{solution}{\begin{solution*}}{{\finishline} \end{solution*}}
\newcommand{\grade}[1]{\hfill{\textbf{($\mathbf{#1}$ points)}}}
\newcommand{\thisdate}{\today}
\newcommand{\thissemester}{\textbf{Rutgers: Spring 2022}}
\newcommand{\thiscourse}{CS 344: Design and Analysis of Computer Algorithms} 
\newcommand{\thishomework}{Number} 
\newcommand{\thisname}{Name} 
\newcommand{\thisextension}{Yes/No} 

\headheight 40pt              
\headsep 10pt
\renewcommand{\headrulewidth}{0pt}
\lhead{\small \textbf{Only for the personal use of students registered in CS 344, Spring 2022 at Rutgers University. Redistribution out of this class is strictly prohibited.}}
\pagestyle{fancy}

\newcommand{\thisheading}{
   \noindent
   \begin{center}
   \framebox{
      \vbox{\vspace{2mm}
    \hbox to 6.28in { \textbf{\thiscourse \hfill \thissemester} }
       \vspace{4mm}
       \hbox to 6.28in { {\Large \hfill Homework \#\thishomework \hfill} }
       \vspace{2mm}
         \hbox to 6.28in { { \hfill Deadline: Tuesday, April 05, 11:59 PM  \hfill} }
       \vspace{2mm}
       \hbox to 6.28in { \emph{Name: \thisname \hfill Extension: \thisextension}}
      \vspace{2mm}}
      }
   \end{center}
   \bigskip
}

%=====================================================
% Some useful MACROS (you can define your own in the same exact way also)
%=====================================================


\newcommand{\ceil}[1]{{\left\lceil{#1}\right\rceil}}
\newcommand{\floor}[1]{{\left\lfloor{#1}\right\rfloor}}
\newcommand{\prob}[1]{\Pr\paren{#1}}
\newcommand{\expect}[1]{\Exp\bracket{#1}}
\newcommand{\var}[1]{\textnormal{Var}\bracket{#1}}
\newcommand{\set}[1]{\ensuremath{\left\{ #1 \right\}}}
\newcommand{\poly}{\mbox{\rm poly}}


%=====================================================
% Fill Out This Part With Your Own Information:
%=====================================================


\renewcommand{\thishomework}{4} %Homework number
\renewcommand{\thisname}{FIRST LAST} % Enter your name here
\renewcommand{\thisextension}{Yes/No} % Pick only one of the two options accordingly

\begin{document}

\thisheading

\subsection*{Homework Policy}
\begin{itemize}
\item If you leave a question completely blank, you will receive 25\% of the grade for that question. This however does not apply to the extra credit questions.
\item You are allowed to discuss the homework problems with other students in the class. \textbf{But you must write your solutions independently.} 
You may also consult all the materials used in this course (video recordings, notes, textbook, etc.) while writing your solution, but no other resources are allowed.
\item Do not forget to write down your name and whether or not you are using one of your two extensions. Submit your homework on Canvas. 
\item  Unless  specified otherwise, you may use any algorithm covered in class as a ``black box'' -- for example you can simply write ``use Prim's or Kruskal's algorithm to find an MST of the input graph in $O(m\log{m})$ time''. 

You are \textbf{strongly encouraged to use graph reductions} instead of designing an algorithm from scratch whenever possible (even when the question does not ask you to do so explicitly). 
\item Remember to always \textbf{prove the correctness} of your algorithms and \textbf{analyze their running time} (or any other efficiency measure asked in the question). 

\item The ``Challenge yourself'' and ``Fun with algorithms''  are both extra credit. These problems are significantly more challenging than the standard problems you see in this course (including lectures, homeworks, and exams). 
As a general rule, only attempt to solve these problems if you  enjoy them. 
\end{itemize}

\finishline




\begin{problem}
A \emph{walk} in a directed graph $G=(V,E)$ from a vertex $s$ to a vertex $t$, is a sequence of vertices $v_1,v_2,\ldots,v_k$ where $v_1 = s$ and $v_k=t$ such that for any $i < k$,  $(v_i,v_{i+1})$ is an edge in $G$. The \emph{length} of a walk
is defined as the number of vertices inside it minus one, i.e., the number of edges (so the walk $v_1,v_2,\ldots,v_k$ has length $k-1$). 

Note that the only difference 
of a walk with a \emph{path} we defined in the course is that a walk can contain the same vertex (or edge) more than once, while a path consists of only distinct vertices and edges. 

Design and analyze an $O(n+m)$ time algorithm that given a directed graph $G=(V,E)$ and two vertices $s$ and $t$, outputs \emph{Yes} if there is a walk from $s$ to $t$ in $G$ \underline{whose length is divisible by five}, and \emph{No} otherwise. 

\grade{25}

\end{problem}

\begin{solution}
The solution to Problem 1 goes here.
\end{solution}


\medskip

\begin{problem}
	We say that an undirected graph $G=(V,E)$ is \textbf{$2$-edge-connected} if we need to remove \emph{at least two} edges from $G$ to make it disconnected. Prove that a 
	graph $G=(V,E)$ is $2$-edge-connected if and only if for every cut $(S,V-S)$ in $G$, there are \emph{at least two cut edges}, i.e., $|{\delta(S)}| \geq 2$. \grade{25}
\end{problem}

\begin{solution}
The solution to Problem 2 goes here.
\end{solution}


\medskip

\begin{problem}
	Given a connected undirected graph $G(V,E)$ with distinct weights $w_e$ on each edge $e \in E$, a \emph{maximum} spanning tree of $G$ is a spanning tree of $G$ with maximum total weight of edges\footnote{This is exactly the opposite 
	of the minimum spanning tree (MST) problem we studied in the lectures.}. Design an $O(m\log{m})$ time algorithm for finding a maximum spanning tree of any given graph. 
	\grade{25}
	
	\emph{Hint:} You may want to use graph reductions, but remember that you are \underline{not} allowed to have \emph{negative} weights on edges of graphs (for now at least). 
\end{problem}

\begin{solution}
The solution to Problem 3 goes here.
\end{solution}


\medskip


\begin{problem}
Let $G=(V,E)$ be an undirected connected graph with maximum edge weight $W_{\mathrm{max}}$. Prove that if an edge with weight $W_\mathrm{max}$ appears in \underline{some} MST of $G$, then \underline{all} MSTs of $G$ contain an edge
 with weight $W_{\mathrm{max}}$. \grade{25}
\end{problem}

\begin{solution}
The solution to Problem 4 goes here.
\end{solution}


\smallskip

\begin{challenge}
	A \textbf{bottleneck spanning tree (BST)} of a undirected connected graph $G=(V,E)$ with positive weights $w_e$ over each edge $e$, is a spanning tree $T$ of $G$ such that the weight of maximum-weight edge in $T$ is minimized 
	across all spanning trees of $G$. In other words, if we define the cost of $T$ as $\max_{e \in T} w_e$, a BST has minimum cost across all spanning trees of $G$. Design and analyze an $O(n+m)$ time algorithm for finding a BST of a given graph. 
	  \grade{+10}
\end{challenge}

\smallskip

\begin{fun}
	Let us go back to the bottleneck spanning tree (BST) problem defined above. 
	\begin{enumerate}
		\item[(a)] Prove that any MST of any graph $G$ is also a BST. Use this to obtain an $O(m\log{m})$ time algorithm for the BST problem (notice that this is  slower than the algorithm from the previous question). 
		
		\grade{+8}
		\item[(b)] Give an example of a BST of some graph $G$ which is \emph{not} an MST in $G$. \grade{+2}
	\end{enumerate}
 \end{fun}


\end{document}




