\documentclass{article}

\setlength{\headsep}{0.75 in}
\setlength{\parindent}{0 in}
\setlength{\parskip}{0.1 in}

%=====================================================
% Add PACKAGES Here (You typically would not need to):
%=====================================================

\usepackage{xcolor}
\usepackage[margin=1in]{geometry}
\usepackage{amsmath,amsthm,amssymb}
\usepackage{fancyhdr}
\usepackage{enumitem}
\usepackage{graphicx}
\usepackage{xspace}
\usepackage{subcaption}
%=====================================================
% Ignore This Part (But Do NOT Delete It:)
%=====================================================

\theoremstyle{definition}
\newtheorem{problem}{Problem}
\newtheorem*{fun}{Fun with Algorithms}
\newtheorem*{challenge}{Challenge Yourself}
\def\fline{\rule{0.75\linewidth}{0.5pt}}
\newcommand{\finishline}{\begin{center}\fline\end{center}}
\newtheorem*{solution*}{Solution}
\newenvironment{solution}{\begin{solution*}}{{\finishline} \end{solution*}}
\newcommand{\grade}[1]{\hfill{\textbf{($\mathbf{#1}$ points)}}}
\newcommand{\thisdate}{\today}
\newcommand{\thissemester}{\textbf{Rutgers: Spring 2022}}
\newcommand{\thiscourse}{CS 344: Design and Analysis of Computer Algorithms} 
\newcommand{\thishomework}{Number} 
\newcommand{\thisname}{Name} 
\newcommand{\thisextension}{Yes/No} 

\headheight 40pt              
\headsep 10pt
\renewcommand{\headrulewidth}{0pt}
\lhead{\small \textbf{Only for the personal use of students registered in CS 344, Spring 2022 at Rutgers University. Redistribution out of this class is strictly prohibited.}}
\pagestyle{fancy}

\newcommand{\thisheading}{
   \noindent
   \begin{center}
   \framebox{
      \vbox{\vspace{2mm}
    \hbox to 6.28in { \textbf{\thiscourse \hfill \thissemester} }
       \vspace{4mm}
       \hbox to 6.28in { {\Large \hfill Homework \#\thishomework \hfill} }
       \vspace{2mm}
         \hbox to 6.28in { { \hfill Deadline: Tuesday, May 03, 11:59 PM  \hfill} }
      \vspace{2mm}
       \hbox to 6.28in { \emph{Name: \thisname \hfill Extension: \thisextension}}
      \vspace{2mm}}
      }
   \end{center}
   \bigskip
}

%=====================================================
% Some useful MACROS (you can define your own in the same exact way also)
%=====================================================


\newcommand{\ceil}[1]{{\left\lceil{#1}\right\rceil}}
\newcommand{\floor}[1]{{\left\lfloor{#1}\right\rfloor}}
\newcommand{\prob}[1]{\Pr\paren{#1}}
\newcommand{\expect}[1]{\Exp\bracket{#1}}
\newcommand{\var}[1]{\textnormal{Var}\bracket{#1}}
\newcommand{\set}[1]{\ensuremath{\left\{ #1 \right\}}}
\newcommand{\poly}{\mbox{\rm poly}}


%=====================================================
% Fill Out This Part With Your Own Information:
%=====================================================


\renewcommand{\thishomework}{5} %Homework number
\renewcommand{\thisname}{FIRST LAST} % Enter your name here
\renewcommand{\thisextension}{Yes/No} % Pick only one of the two options accordingly

\begin{document}

\thisheading

\subsection*{Homework Policy}
\begin{itemize}
\item If you leave a question completely blank, you will receive 25\% of the grade for that question. This however does not apply to the extra credit questions.
\item You are allowed to discuss the homework problems with other students in the class. \textbf{But you must write your solutions independently.} 
You may also consult all the materials used in this course (video recordings, notes, textbook, etc.) while writing your solution, but no other resources are allowed.
\item Do not forget to write down your name and whether or not you are using one of your two extensions. Submit your homework on Canvas. 
\item  Unless  specified otherwise, you may use any algorithm covered in class as a ``black box'' -- for example you can simply write ``use Ford-Fulkerson's algorithm to find a maximum flow of the input network in $O(m \cdot F)$ time''. 

You are \textbf{strongly encouraged to use graph reductions} instead of designing an algorithm from scratch whenever possible (even when the question does not ask you to do so explicitly). 
\item Remember to always \textbf{prove the correctness} of your algorithms and \textbf{analyze their running time} (or any other efficiency measure asked in the question). 

\item The ``Challenge yourself'' and ``Fun with algorithms''  are both extra credit. These problems are significantly more challenging than the standard problems you see in this course (including lectures, homeworks, and exams). 
As a general rule, only attempt to solve these problems if you  enjoy them. 
\end{itemize}

\finishline

\begin{problem}
	You are given a weighted undirected graph $G=(V,E)$ with integer weights $w_e \in \set{1,2,\ldots,W}$ on each edge $e$, where $W \leq 1000$. 
	Given two vertices $s,t \in V$, the goal is to find the minimum weight path (or shortest path) from $s$ to $t$. Recall that Dijkstra's algorithm solves this problem in $O(n+m\log{m})$ time even if we do not have the condition that $W \leq 1000$. 
	However, we now want to use this extra condition to design an even faster algorithm. 
	
	
	Design and analyze an algorithm to find the minimum weight (shortest) $s$-$t$ path on these restricted weighted graphs in $O(n+m)$ time. \grade{30}
\end{problem}

\bigskip

\begin{solution}
	Solution to Problem 1 goes here. 
\end{solution}
\bigskip

\begin{problem}
	You are given an $n \times n$ matrix and a set of $k$ cells $(i_1,j_1),\ldots,(i_k,j_k)$ on this matrix. We say that this set of cells can \textbf{escape} the matrix if: (1) we can find a path from each cell to any arbitrary \emph{boundary cell} of the matrix 
	(a path is a sequence of \emph{neighboring} cells, namely, top, bottom, left, and right), (2) these paths are all \emph{disjoint},  namely, no cell is used in more than one of these paths. 

Design an $O(n^3)$ time algorithm that given the matrix and the input cells, determines whether these cells can escape the matrix (together) or not\footnote{This (type of) problem is used typically to connect multiple components of a circuit to the power sources that can only be placed at the boundary of the circuit -- disjointness of the paths ensures that the components of the circuit do not interfere with each other.}.
\grade{30}

\end{problem}


\bigskip

\begin{solution}
	Solution to Problem 2 goes here. 
\end{solution}

\bigskip




\begin{problem}
Prove that each of the following problems is NP-hard and for each problem determine whether it is also NP-complete or not -- if the problem is NP-complete, you also need to give a poly-time verifier and argue the correctness of your verifier. 

\begin{enumerate}[label=(\alph*)]
\item \textbf{One-Tenth-Path Problem:} Given an undirected graph $G=(V,E)$, does $G$ contain a path that passes through \emph{at least one tenth}  of the vertices in $G$? \grade{15}

\bigskip

\begin{solution}
	Solution to Problem 3 Part (a) goes here. 
\end{solution}

	\item \textbf{Minus-One 3-SAT Problem:} Given a 3-CNF formula $\Phi$ with $m$ clauses and $n$ variables (in which size of each clause is \emph{at most} $3$), is there an assignment to the variables that satisfies exactly $m-1$ clauses, namely, all minus one of them?  \grade{15}

\bigskip

\begin{solution}
	Solution to Problem 3 Part (b) goes here. 
\end{solution}

	 \item \textbf{Negative-Weight Shortest Path Problem:} Given an undirected graph $G=(V,E)$, two vertices $s,t$ and \emph{negative} weights on the edges, what is the weight of the shortest path from $s$ to $t$? \grade{10}
\end{enumerate}

\bigskip

\begin{solution}
	Solution to Problem 3 Part (c) goes here. 
\end{solution}

You may assume the following problems are NP-hard for your reductions: 
\begin{itemize}
	\item \textbf{Undirected $s$-$t$ Hamiltonian Path:} Given an undirected graph $G=(V,E)$ and two vertices $s,t \in V$, is there a Hamiltonian path from $s$ to $t$ in $G$? (A Hamiltonian path is a path that passes every vertex). 
	\item \textbf{3-SAT Problem:} Given a 3-CNF formula $\Phi$  (where each clause as \emph{at most} $3$ variables), is there an assignment to $\Phi$ that makes it true?
\end{itemize}

\end{problem}

\finishline


\begin{fun}
	You are given a puzzle consists of an $m \times n$ grid of squares, where each
square can be empty, occupied by a red stone, or occupied by a blue stone. The goal of the puzzle
is to remove some of the given stones so that the remaining stones satisfy two conditions: (1) every row contains at least one stone,
and (2) no column contains stones of both colors. 

It is easy to see that for some initial configurations of stones, reaching this goal is impossible. 
We define the Puzzle problem as follows. Given an initial configuration of red and blue stones on an $m \times n$ grid of squares,
determine whether or not the puzzle instance has a feasible solution. 

Prove that the Puzzle problem is NP-complete. \grade{+10}
\end{fun}

\newpage

Consider solving \textbf{at most one} of the following two challenge yourself problems. 

\bigskip

\begin{challenge}[\textbf{I}] 
	The goal of this question is to give a simple proof that there are decision problems that admit \emph{no} algorithm at all (independent of the runtime of the algorithm). 
	
	Define $\Sigma^+$ as the set of all \emph{binary} strings, i.e., $\Sigma^{+} = \set{0,1,00,01,10,11,000,001,\ldots}$. Observe that any decision problem $\Pi$ can be identified by a function $f_{\Pi} : \Sigma^{+} \rightarrow \set{0,1}$. 
	Moreover, observe that any algorithm can be identified with a binary string in $\Sigma^{+}$. Use this to argue that ``number'' of algorithms is ``much smaller'' than ``number'' of decision problems and hence there should be some decision problems
	that cannot be solved by any algorithm. 
	
	\emph{Hint:} Note that in the above argument you have to be careful when comparing ``number'' of algorithms and decision problems: after all, they are both infinity! Use the fact that \emph{cardinality} of the set of real numbers $\mathbb{R}$ is larger than the cardinality of integer numbers $\mathbb{N}$ (if you have never seen the notion of cardinality of an infinite set before, you may want to skip this problem).  \grade{+10}
	
\end{challenge}

\bigskip

\begin{challenge}[\textbf{II}] 
	Recall that in the class, we  focused on \emph{decision} problems when defining NP. Solving a decision problem simply tells us whether a solution to our problem exists or not but  it does not provide that solution when it 
	exists. Concretely, let us consider the 3-SAT problem on an input formula $\Phi$. Solving 3-SAT on $\Phi$ would  tell us whether $\Phi$ is satisfiable or not but will not give us a satisfying assignment when $\Phi$ is satisfiable. 
	What if our goal is to actually find the satisfying formula when one exists? This is called a \emph{search} problem. 
	
	It is easy to see that a search problem can only be ``harder'' than its decision variant, or in other words, if we have an algorithm for the search problem we will obtain an algorithm for the decision problem as well. Interestingly, the converse 
	of this is also true for all NP problems and we will prove this in the context of the 3-SAT problem in this problem. In particular, we reduce the 3-SAT-SEARCH problem (the problem of finding a satisfying assignment to a 3-CNF formula) 
	to the 3-SAT (decision) problem (the problem of deciding whether a 3-CNF formula has a satisfying assignment or not). 
	
	Suppose you are given, as a black-box, an algorithm $A$ for solving 3-SAT (decision) problem that runs in polynomial time. Use $A$ to design a poly-time algorithm that given a 3-CNF formula $\Phi$, either outputs $\Phi$ is not satisfiable or 
	\emph{finds} an assignment $x$ such that $\Phi(x) = True$. \grade{+10}
\end{challenge}



\end{document}




