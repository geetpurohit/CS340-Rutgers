\documentclass{article}

\setlength{\headsep}{0.75 in}
\setlength{\parindent}{0 in}
\setlength{\parskip}{0.1 in}

%=====================================================
% Add PACKAGES Here (You typically would not need to):
%=====================================================

\usepackage[margin=1in]{geometry}
\usepackage{amsmath,amsthm}
\usepackage{fancyhdr}
\usepackage{enumitem}
\usepackage{graphicx}
%=====================================================
% Ignore This Part (But Do NOT Delete It:)
%=====================================================

\theoremstyle{definition}
\newtheorem{problem}{Problem}
\newtheorem*{fun}{Fun with Algorithms}
\newtheorem*{challenge}{Challenge Yourself}
\def\fline{\rule{0.75\linewidth}{0.5pt}}
\newcommand{\finishline}{\vspace{-15pt}\begin{center}\fline\end{center}}
\newtheorem*{solution*}{Solution}
\newenvironment{solution}{\begin{solution*}}{{} \end{solution*}}
\newcommand{\grade}[1]{\hfill{\textbf{($\mathbf{#1}$ points)}}}
\newcommand{\thisdate}{\today}
\newcommand{\thissemester}{\textbf{Rutgers: Spring 2022}}
\newcommand{\thiscourse}{CS 344: Design and Analysis of Computer Algorithms} 
\newcommand{\thishomework}{Number} 
\newcommand{\thisname}{Name} 
\newcommand{\thisextension}{Yes/No} 

\headheight 40pt              
\headsep 20pt
\renewcommand{\headrulewidth}{0pt}
\lhead{\small \textbf{Only for the personal use of students registered in CS 344, Spring 2022 at Rutgers University. Redistribution out of this class is strictly prohibited.}}
\pagestyle{fancy}

\newcommand{\thisheading}{
   \noindent
   \begin{center}
   \framebox{
      \vbox{\vspace{2mm}
    \hbox to 6.28in { \textbf{\thiscourse \hfill \thissemester} }
       \vspace{8mm}
       \hbox to 6.28in { {\Large \hfill  Midterm Exam \#\thishomework \hfill} }
       \vspace{2mm}
         \hbox to 6.28in { { \hfill ~ February 24, 2022 \hfill} }
       \vspace{4mm}
       \hbox to 6.28in { \emph{Name: \underline{\thisname} \hfill NetID: \underline{\thisextension}}}
      \vspace{2mm}}
      }
   \end{center}
   \bigskip
}

%=====================================================
% Some useful MACROS (you can define your own in the same exact way also)
%=====================================================


\newcommand{\ceil}[1]{{\left\lceil{#1}\right\rceil}}
\newcommand{\floor}[1]{{\left\lfloor{#1}\right\rfloor}}
\newcommand{\prob}[1]{\Pr\paren{#1}}
\newcommand{\expect}[1]{\Exp\bracket{#1}}
\newcommand{\var}[1]{\textnormal{Var}\bracket{#1}}
\newcommand{\set}[1]{\ensuremath{\left\{ #1 \right\}}}
\newcommand{\poly}{\mbox{\rm poly}}

\newcommand{\totalsum}{\textnormal{\texttt{TOTAL-SUM}}}
\newcommand{\maxalg}{\textnormal{\texttt{MAX-ALG}}}
\newcommand{\minalg}{\textnormal{\texttt{MIN-RAND-ALG}}}

%=====================================================
% Fill Out This Part With Your Own Information:
%=====================================================


\renewcommand{\thishomework}{1} %Exam number -- Do NOT change
\renewcommand{\thisname}{name} % Your name
\renewcommand{\thisextension}{netid} % Write your NetID here

\begin{document}

\thisheading

\vspace{-0.5cm}
\subsection*{Instructions}

\begin{enumerate}
	\item Do not forget to write your name and NetID above, and to sign Rutgers honor pledge below. 
	\item The exam contains $4$ problems worth $100$ points in total \emph{plus} one {extra} credit problem worth $10$ points. 
	\item This is a take-home exam. You have exactly 24 hours to finish the exam, starting from Thursday, February 24, 9am EST until Friday, February 25, 9am EST.
	\item The exam should be done \textbf{individually} and you are not allowed to discuss these questions with anyone else. This includes asking any questions or clarifications
	regarding the exam from other students or posting them publicly on Piazza (\textbf{any inquiry should be posted privately on Piazza}). You may however consult all
	the materials used in this course (video lectures, notes, textbook, etc.) while writing your solution, but \textbf{no other resources are allowed}.

	\item Remember that you can leave a problem (or parts of it) entirely blank and receive $25\%$ of the grade for that problem (or part). However, this should not  discourage you from attempting a problem if you think 
	you know how to approach it as you will receive partial credit more than $25\%$ if you are on the right track. But keep in mind that if you simply do not know the answer, writing a very wrong answer may lead to $0\%$ credit.
	
	The only \textbf{exception} to this rule is the extra credit problem: you do {not get any credit for leaving the extra credit problem blank}, and there is almost no partial credit on that problem.
	
	\item \textbf{You should always prove the correctness of your algorithm and analyze its runtime.} Also, as a general rule, avoid using complicated pseudo-code and instead explain your algorithm in English. 
	\item You may use any algorithm presented in the class or homeworks as a building block for your solutions. 
\end{enumerate}

\finishline

\paragraph{Rutgers honor pledge:} 

\begin{quote}
\emph{On my honor, I have neither received nor given any unauthorized assistance on this
examination.} 
\end{quote}
\hfill{Signature:\underline{~~~~~~~~~~~~~~~~~~~~~~~~~~~~~~~~~~~~~~~~~~~~~}}

\bigskip

\begin{center}
\begin{tabular}{|c|r|c|}
\hline
Problem. \# & Points & Score \\ \hline\hline
$1$ & 25 & ~~~~~~~~~~~\\  \hline
$2$ & 25 & \\ \hline
$3$ & 25 & \\ \hline
$4$ & 25 & \\ \hline
$5$ & +10 & \\ \hline
Total & $100 + 10$ & \\ \hline
\end{tabular}
\end{center}

\newpage

\begin{problem}\label{basics}~
\begin{enumerate}[label=(\alph*)]
	\item  Determine the \emph{strongest} asymptotic relation between the functions 
	\[
	f(n) = 2^{\sqrt{n}} \quad \text{and} \quad g(n) = n^4,
	\]
	i.e., whether $f(n) = o(g(n))$, $f(n) = O(g(n))$, $f(n) = \Omega(g(n))$, $f(n) = \omega(g(n))$, or $f(n) = \Theta(g(n))$. 
	Remember to prove the correctness of your choice. \grade{15}

\smallskip
\begin{solution}
	% Write your solution starting from next line: 
\end{solution}

	\newpage
	\item Use the \emph{recursion tree} method to solve the following recurrence $T(n)$ by finding the \emph{tightest} function $f(n)$ such that $T(n) = O(f(n))$.   \grade{10}
	\begin{align*}
		T(n) &\leq 3 \cdot T(n/2) + O(n^2) 
	\end{align*} 
	
	
\smallskip
\begin{solution}
	% Write your solution starting from next line: 
\end{solution}


	\vspace*{\fill}
\end{enumerate}
\end{problem}

\newpage


\begin{problem}\label{induction}
	Consider the algorithm below for finding the sum of numbers in an array $A$ of size $n \geq 1$.  
	
	\medskip
	
	$\totalsum (A[1:n])$: \vspace{-0.4cm}
	\medskip
	\begin{enumerate}
		\item If $n=1$, return $A[1]$. 
		\item Otherwise, return $\totalsum(A[1:n-1])$ + $A[n]$. 
	\end{enumerate}

	We analyze \totalsum~ in this question. 
	\begin{enumerate}[label=(\alph*)]
		\item Use \textbf{induction} to prove the correctness of this algorithm. \grade{15}
		
		
\smallskip
\begin{solution}
	% Write your solution starting from next line: 
\end{solution}
		\newpage
		\item Write a recurrence for this algorithm and solve it to obtain a tight upper bound on the worst case runtime of this algorithm. You can use any method you like for solving this recurrence. 		\grade{10}
		
		
\smallskip
\begin{solution}
	% Write your solution starting from next line: 
\end{solution}

	\end{enumerate}
\end{problem} 

\newpage

\begin{problem}\label{sort}
	You are given an \emph{unsorted} array $A[1:n]$ of $n$ distinct positive integers. Design an algorithm that in $O(n)$ worst-case time finds the largest number in $\set{1,\ldots,n+1}$ that is missing from $A$. 
	\grade{25}
	
	Remember to \emph{separately} write your algorithm (\textbf{10 points}), the proof of correctness (\textbf{10 points}), and runtime analysis (\textbf{5 points}). 
	
	\medskip
	\emph{Example.} For $n=8$ and $A=[9,8,70,30,2,7,4,5]$ the correct answer is $6$ (because the largest number in $\set{1,\ldots,9}$ which is missing from $A$ is $6$). 

\end{problem}


\smallskip
\begin{solution}
	% Write your solution starting from next line: 
\end{solution}

\newpage

\begin{problem}\label{DP}
We want to purchase an item of price $M$ and for that we have a collection of $n$ different coins in an array $C[1:n]$ where coin $i$ has value $C[i]$ (we only have one copy of each coin).  
	Our goal is to purchase this item using the \emph{smallest} possible number of coins or outputting that the item cannot be purchased with these coins. 
	Design a \textbf{dynamic programming} algorithm for this problem with worst-case runtime of $O(n \cdot M)$. \grade{25}
	
	Remember to \emph{separately} write your specification of recursive formula in plain English (\textbf{7.5 points}), your recursive solution for the formula and its proof of correctness (\textbf{7.5 points}), the algorithm using either memoization 
	or bottom-up dynamic programming (\textbf{5 points}), and runtime analysis (\textbf{5 points}). 
	
	\emph{Example:} 
	\begin{itemize}
		\item Given $M=15$, $n=7$, and $C = [4,9,3,2,7,5,6]$, the correct answer is $2$ by picking $C[2]=9$ and $C[7]=6$ which add up to $15$. 
		\item Given $M=11$, $n=4$, and $C = [4,3,5,9]$, the correct answer is that `the item cannot be purchased' as no combination of these coins adds up to a value of $11$ (recall that we can only use each coin once). 
	\end{itemize}
\end{problem}

\smallskip
\begin{solution}
	% Write your solution starting from next line: 
\end{solution}

\newpage
\begin{problem}[\textbf{Extra credit}]\label{extra}
	You are given an \emph{unsorted} array $A[1:n]$ of $n$ distinct numbers with the promise that $A[1] > A[2]$ and $A[n] > A[n-1]$. 
	Design an algorithm that in $O(\log{n})$ worst case runtime finds an index $i \in \set{2,\ldots,n-1}$ such that $A[i]$ is smaller than both $A[i-1]$ and $A[i+1]$. If there are multiple indices with this property, your algorithm can output one of them. 
	\grade{+10}

\emph{Example:} Given $A=[3,2,7,8,1,4,5,6]$, one right answer is to output $A[2]=2$, and another one is $A[5]=1$. 
\end{problem}

\smallskip
\begin{solution}
	% Write your solution starting from next line: 
\end{solution}

\newpage
\subsection*{Extra Workspace}


\end{document}





